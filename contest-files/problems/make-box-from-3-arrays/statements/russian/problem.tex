\begin{problem}{Волшебная клетка}{стандартный ввод}{стандартный вывод}{1 секунда}{256 мегабайт}

Три волшебника победили зло и теперь им необходимо запереть его в клетку. У каждого мага есть набор волшебных прутиков, длина каждого такого прутика является целых положительным числом.

Чтобы создать клетку, каждый из трёх магов должен выбрать один из своих прутиков, обозначим их длины за $a$, $b$ и $c$, соответственно. Клетка будет представлять из себя параллелепипед высотой $a$, шириной $b$ и глубиной $c$. Объём этой клетки будет равен $a \times b \times c$.

Чтобы зло не могло разрушить клетку, её объём должен быть нечётным, но при этом, делиться нацело на 3.

Вычислите количество способов построить желаемую клетку.

\InputFile
Ввод содержит три строки, каждая из которых описывает набор прутиков очередного мага.

В начале строки содержится число $n$~--- количество прутиков ($1 \leq n \leq 30\,000$), а затем $n$ целых чисел $l_{i}$~--- длины прутиков ($1 \leq l_{i} \leq 10\,000$).

\OutputFile
В единственную строку выведите целое число~--- количество способов собрать клетку.

\Examples

\begin{example}
\exmpfile{example.01}{example.01.a}%
\exmpfile{example.02}{example.02.a}%
\end{example}

\end{problem}

